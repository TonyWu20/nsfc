
鉴于全球气候变化与环境污染问题日益严峻,我国提出了``双碳''目标(2030年碳达峰、2060年碳中和),亟需发展高效清洁能源技术以替代传统化石能源体系\cite{Chen_2024,Xia_2022}。
在此背景下,电催化技术作为连接可再生能源与高附加值化学品的核心桥梁,是应对未来清洁能源技术需求的关键选择\cite{Zhu_2021}。
其战略地位主要体现在以下三方面:能源转换高效性、环境治理靶向性和技术集成灵活性。
首先,电催化反应(如电解水制氢、二氧化碳还原(\ce{CO2}RR))可直接利用风电、光伏等间歇性绿色电源驱动,将电能转化为氢能或碳基燃料(如甲醇和甲酸),突破传统能源存储的时空限制,实现``绿电——绿氢——绿化工''的全链条闭环\cite{Jiao_2015}。
其次,电催化技术通过原子级精准调控反应路径,可针对性地将\ce{CO2}、\ce{NO_{X}}等环境污染物转化为高价值化学品\cite{Liu_2022},兼具碳减排与资源化双重功能。
小分子基电催化是指利用电催化技术,通过催化剂与电场的协同作用,促进小分子(如
\ce{H2O}、\ce{CO2}、\ce{N2}、\ce{CH3OH}、\ce{C2H5OH}、尿素、肼等)的氧化或还原反应,实现高效、绿色的化学转化或能源转换过程\cite{Peng_2023,Wang_2022,Zhang_2016,Quan_2021}。

然而,当前小分子电催化技术仍面临两大核心挑战:反应速度缓慢,需要高效催化剂;催化剂成本高昂。
以 \ce{CO2}RR 为例,其多电子转移过程涉及复杂的中间体吸附和脱附行为,导致过电位高和选择性差两大关键问题\cite{Liu_2022}。
在该类小分子电催化中,目前常见的工业和商用催化剂的活性成分都以铂(Pt)、钯(Pd)、钌(Ru)等贵金属为主\cite{Du_2019,Nie_2015},虽然活性优异,但其稀缺性与高成本制约了大规模应用。
尽管近年来利用铁(Fe)、钴(Co)、镍(Ni)、铜(Cu)等相对廉价的过渡金属及碳基、氮化物等非金属材料作为小分子基电化学反应的催化剂的研究日益增多,贵金属催化剂仍因其独有的优势在小分子电催化中占据主导地位。
首先,贵金属元素因其独特的 d 电子结构,对反应中间体(如*H、*OOH、*COOH)往往具有适中的吸附强度,可显著降低反应活化能。
以酸性析氢反应(acidic
HER)为例,Pt 基催化剂在酸性 HER 中表现出接近热力学平衡的过电位(\textasciitilde10
mV@10
mA/cm\textsuperscript{2}),远超过渡金属硫化物或碳基材料\cite{Jiao_2015,Wei_2024} 。
其次,贵金属在电催化反应环境下具有本征的高化学稳定性,更有利于催化剂在实际工况中的部署,降低因催化剂失活而带来的额外成本。
而且,其稳定性还能通过纳米工程手段进一步增强。
例如,张涛等人通过 Ru 基催化剂的晶格应变工程,将 OER 过电位降低至270
mV,且稳定性超过100小时,凸显贵金属在苛刻工况下的耐久性\cite{Hou_2023}。
相比之下,过渡金属催化剂常因活性位点易氧化或溶解而导致稳定性不足,而非金属材料(如石墨烯)则受限于导电性与活性位点密度低等问题\cite{Maiti_2021}。
因此,贵金属催化剂仍是当前实现高效小分子电催化的热门选择。

与此同时,贵金属的稀缺性与高成本迫使研究者寻找进一步调控活性、提升原子利用率的方法。
根据Sabatier原理,理想的催化剂需对反应中间体具有适中的吸附强度:过强会导致活性位点中毒,过弱则会阻碍反应动力学\cite{Hu_2021}。
因此,需要对贵金属催化剂的电子结构进行精准的调控,以使其表面吸附强度达到最优水平,降低反应活化能势垒。
为实现这一目标,应力工程(strain
engineering)作为一种精准调控催化剂表面电子结构与几何构型的关键技术,逐渐成为研究热点。
与其他纳米工程手段(如尺寸控制、原子掺杂或合金化)相比,应力工程可通过对晶格常数的微调实现催化性能的定量优化\cite{Hammer_2000}。
接下来,我们将从应力工程的具体定义与优势、历史发展、近期进展及现存挑战等方面,系统阐述应力工程在贵金属纳米催化剂中的应用及亟待进一步探索的科学问题。

应力工程通常指通过引入晶格应变(lattice
strain)或对材料施加机械应力,调控材料表面原子间距,从而改变其电子结构与催化性能的技术。
晶格应变的定义如下:

$$
	s = \left( \frac{a_{\text{shell}} - a_{\text{bulk}}}{a_{\text{bulk}}} \right) \times 100\%
$$

其中,$a_\text{shell}$为表面原子间距,$a_\text{bulk}$为本体材料晶格常数。
相较于其他纳米工程手段(如尺寸调控或形貌设计),应力工程具有显著优势。
首先,其调控具有连续性与精细性,可通过调整应变幅度实现催化活性的线性优化\cite{Hammer_2000,Strasser_2010};其次,应变可通过核壳结构、缺陷工程、合金化等多种机制引入,具有高度的设计灵活性\cite{Chen_2024,Hou_2023,Maiti_2021}。
例如,核壳结构通过内核与外壳的晶格失配产生表面应变\cite{Strasser_2010,Xue_2021,Bu_2016,Moseley_2015,E_2018},而缺陷(如晶界或位错)则通过局部原子排列畸变改变吸附位点性质\cite{Zhang_2021,Zhao_2021}。
此外,纳米颗粒尺寸的变化亦可诱导应变:较小的颗粒因表面原子的自发压缩倾向,而呈现被压缩的晶格参数\cite{Beyerlein_2012}。
这些特性使应力工程成为优化贵金属催化剂的有效策略,尤其在碳中性能源转化领域展现出广阔前景\cite{Chen_2024}。

应力工程的理论基础可追溯至 20 世纪末的电子结构研究。
1983 年,Taylor 等人首次发现Pt基合金的氧还原反应(ORR)活性与原子间距存在相关性,但当时仅基于几何模型,未阐明应变对吸附特性的影响机制\cite{Jalan_1983}。
1998 年,Nørskov 团队提出 d 带中心模型,建立了表面应变与催化活性之间的定量关系\cite{Mavrikakis_1998,Hammer_1995}。
该模型指出,拉伸应变使晚期过渡金属 d 带中心能级上移,增强与吸附物的相互作用,而压缩应变则下移 d 带中心能级,削弱吸附强度。
这一理论为后期应变调控提供了重要指导。
如 Mukerjee 等人通过调控 PtCo 纳米催化剂的成分引入压缩应变,其 ORR 动力学速率常数较未应变样品提升52\%\cite{Jia_2015}。
2015年,Bard 团队利用形状记忆合金基底对 Pt 纳米膜施加弹性应变,发现5\%的压缩应变使 ORR 活性提升90\%,而拉伸应变则降低40\%\cite{Du_2015}。

然而,随着纳米材料结构复杂性的增加,应变效应的解释面临新的挑战。
2016年,郭少军等人发现 Pt 壳层的大幅拉伸应变削弱了 Pt-O 键合强度,从而提升了 ORR 活性。
这一发现与``拉伸应变—— d 带中心上移——增强吸附强度''的传统结论相悖\cite{Bu_2016} 。
又如,2022年王功名与洪勋等人研究表明,4\%的拉伸应变虽降低 d 带中心能级,但仍略微增强氢吸附自由能($\Delta G_{\textrm{H}^{*}}$),说明单一电子效应不足以解释所有实验现象\cite{Wu_2022}。
另外,近期已有多项研究表明,除了作用于催化剂的应力作用,反应中间体的表面吸附也会带来催化剂表面的局部变型,从而产生局域的应力作用,且该局域应力作用导致的电子结构变化不可被轻易忽视\cite{Shu_2001,Pala_2004,Francis_2015,Khorshidi_2018}。
此类结果说明,我们需要重新审视应变对电子结构与几何构型的协同影响,及过往理论模型的阐释。
以黄小青等人对 PtPb/Pt 核壳纳米碟片的研究为例,他们发现拉伸应变导致的 Pt-Pt 间距增大可减少桥位点数量,从而抑制甲醇重构反应中催化剂活性位点 CO 中毒的现象\cite{E_2018}。
这表明,若要精准表征应力对贵金属催化材料表面活性的调控效果,除电子效应外,应变引起的几何位点变化同样不可忽视。

近年来,应力工程的研究焦点逐渐转向多机制协同作用与新型材料体系的开发。
在合金体系方面,曾杰团队通过原子级构建拉伸应变的 PdFe 合金表面,发现应变与配体效应共同优化了氧中间体的吸附强度,使 ORR 质量活性提升8.7倍\cite{Li_2020}。
类似地,蔡彬团队制备的 Ru-Au 双金属气凝胶通过微应变调控 d 带结构与表面电子密度,显著提升碱性条件下的 HER 活性\cite{Wei_2024}。
这些研究表明,应变与合金组分间的电荷再分布可产生协同催化效应。
在核壳结构领域,郭少军团队通过精确调控 Ir/Pd 核壳纳米颗粒的拉伸应变,使 HER 过电位降低至17
mV(10
mA/cm\textsuperscript{2}),并证实适度的拉伸应变可减弱H吸附强度,突破传统 d 带中心模型的预测范围\cite{Guo_2023a}。
此外,缺陷工程为应变引入提供了新思路。
郭少军团队利用晶界处的局部应变调控 Pt 表面活性,发现高指数晶面因应变诱导的电子离域化表现出更优的甲醇氧化活性\cite{Zhao_2021}。
与此同时,高传博团队开发的 Au-Ag-Pd 合金纳米线通过高拉伸应变优化了生物质醇类的电氧化路径,展现出优异的催化选择性\cite{Zhang_2021}。
这些进展凸显了应变调控手段的多样性与创新性。

尽管应力工程已取得显著成果,其理论基础与实际应用仍面临多重挑战。
首先,初始提出的 d 带中心模型在解释复杂体系时存在局限性。
例如,早期过渡金属的 d 带填充度较低,其应变响应趋势与晚期过渡金属相反,需结合轨道杂化与电荷转移效应综合分析\cite{Schnur_2010}。
其次,应变效应的各向异性性质尚未得到充分重视。
陈煜团队近期发现,\ce{Pd3Pb} 纳米线的拉伸应变会定向改变 Pd-Pd 轨道的杂化程度,从而选择性增强乙醇氧化反应的 C-C 键断裂效率\cite{Miao_2024}。
这一发现表明,应变方向与晶格畸变模式对催化路径具有决定性影响,而现有研究多基于均匀或单、双轴应变假设,亟需发展针对各向异性应变进行分析的理论模型。
此外,在合金结构中,应变与电荷再分布的贡献需要进一步辨析。
例如,在 Au-Ag-Pd 合金纳米线中,拉伸应变与组分间的电子转移共同调控醇类氧化活性,但二者相对权重尚不明确\cite{Zhang_2021}。
值得关注的是,近期研究发现应力工程可突破反应物之间的吸附能线性关联关系(linear
scaling
relationship)的限制。
例如 Peterson 等人提出``本征应力''(eigenstress)概念,通过局部应变差异化调控不同中间体的吸附能,从而打破吸附能的线性关联,使得精细化独立调控反应路径中的反应能垒成为可能,可大大提高催化剂的理论活性上限\cite{Khorshidi_2018}。
最后,应变结构的稳定性是实际应用的瓶颈。
崔屹团队通过可调电池电极材料实现了催化应变的连续调控,为动态应变工程奠定了基础\cite{Wang_2016}。
未来研究需进一步探索催化反应进行时的应变弛豫机制\cite{Chattot_2021},并通过高通量计算筛选高稳定性应变材料。
这些研究为应变工程的理论深化提供了新视角。

以上回顾表明,应力工程通过精准调控贵金属催化剂的电子与几何结构,为优化并突破吸附能线性关联带来的活性限制提供了新途径。
从早期的 d 带中心模型到近年来的多机制协同研究,该领域已实现从理论探索到实际应用的跨越。
然而,复杂体系的应变响应机制、各向异性效应及稳定性问题仍需深入探究。

本项目正是在这样的研究背景下提出的。
为了系统化、理性化、定量化地分析与预测应力对贵金属电催化剂催化活性的调控效果,亟需建立贵金属应力行为数据库,将实验表征的应力数据与第一性原理计算建模相结合,分析预测贵金属催化材料电子结构在应力作用下的变化趋势与特征,从而为其在小分子基电催化中的活性提供理论预测。

这一数据库的构建将具有以下重要科学意义与关键问题解决能力:
首先,现有理论模型对复杂应力效应的解释存在局限性,且目前依然停留在定性分析,缺乏根据电子结构参数变化定量预测吸附强度变化趋势的能力。
尽管 d 带中心理论在多数情况下可定性解释压缩/拉伸应变对吸附强度的影响\cite{Hammer_1995},但在实际体系中,应力作用常伴随配体效应、轨道杂化或局部几何畸变,导致电子结构变化超出传统理论假设。
以王功名与洪勋团队研究为例,超薄贵金属纳米片的平面应变虽显著提升HER活性,但 d 带中心变化与吸附自由能($\Delta G_{H^{*}}$)的关联呈现非线性特征\cite{Wu_2022}。
此外,高传博团队在 Au-Ag-Pd 合金纳米线中观察到应变与电荷转移的协同作用,难以通过单一理论模型区分二者贡献\cite{Zhang_2021}。
通过构建数据库,可整合不同应力条件下材料的电子结构参数(如 d 带中心、电荷密度分布)与吸附能数据,结合机器学习算法挖掘隐藏的量化规律,从而建立超越传统 d 带理论的多参数预测模型,探寻具体电子结构与几何结构参数变动对应的吸附强度变化关系,为解决``应力-吸附''关系的复杂性问题提供数据驱动的新思路。
其次,实验研究中材料筛选的低效性亟待突破。
目前应力调控研究多依赖经验性探索,如郭少军团队通过试错法优化Ir/Pd核壳结构的拉伸应变以提升HER活性\cite{Guo_2023a},
但此类方法难以系统覆盖多元合金或复杂缺陷体系\cite{Yao_2023}。
若以第一性原理计算为主导,在数据库中系统模拟不同贵金属元素(如Pt、Pd、Ru)及其组合在应变作用下的催化性能,则可实现``以点扩面''的高效筛选。
比如,基于现有PtNi合金的拉伸应变研究\cite{Zhao_2020},数据库可快速预测PdFe或RuAu体系的应力响应趋势,指导实验团队优先验证高潜力候选材料。
这种``理论模拟先行、实验验证跟进''的模式,将显著减少试错成本,加速新型应变催化剂的开发进程。
最后,高通量计算与数据库结合可揭示应力对催化选择性的调控机制。
例如,通过分子动力学手段,模拟小分子(如\ce{CO2}、\ce{NH3})在所有数据库收录的应变材料表面的吸附路径,并关联数据库中的活性位点参数,可定量解析应力对反应选择性的影响。
最近,辛洪良等团队利用机器学习结合应变调控,成功打破硝酸盐还原反应中吸附能的线性标度关系\cite{Gao_2022},表明数据库支持下的多尺度模拟能精准预测应变对特定反应路径的调控效果。
此外,针对碳中和小分子电催化耦合(如二氧化碳与含氮分子转化\cite{Liu_2022}),数据库可整合应力作用下贵金属表面对不同中间体的吸附偏好性数据,为设计高选择性催化剂提供理论依据。
这种数据驱动的分析模式,不仅能够总结普适性规律,还可为特定应用场景定制应变优化策略。

综上所述,贵金属应力行为数据库的建立,将从理论突破、材料筛选、反应优化三个维度推动应力工程研究的范式革新。其核心价值在于将分散的实验现象与理论预测转化为结构化知识体系,为理性设计高性能电催化剂提供坚实的科学基础。未来,结合原位表征技术与人工智能算法,这一数据库有望成为连接纳米材料设计与清洁能源应用的关键桥梁。

在本研究项目的前期准备中,我们通过对过渡金属及贵金属低指数面晶面及高指数面的理论计算建模,发现了应力作用对电子结构调控具有方向性的重要证据\cite{Wu_2020},并初步提出了预测高指数面纳米颗粒表面活性与稳定性的参数指标\cite{Wu_2021}。对于我们的研究而言,这将是深入探寻应力作用对贵金属电子结构调控机理以及系统性归纳整理贵金属应力行为数据的序幕。

\begin{REF}
	\subsection*{参考文献}
	\vspace{-50pt}
	\bibliography{main.bib}%参考文献
\end{REF}

\newpage%自己判断是否需要
