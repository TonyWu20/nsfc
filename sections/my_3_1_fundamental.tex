
	\providecommand{\bibauthor}[1]{#1}
	\providecommand{\bibeditor}[1]{#1}
	\providecommand{\bibtranslator}[1]{#1}
	\providecommand{\bibtitle}[1]{#1}
	\providecommand{\bibbooktitle}[1]{#1}
	\providecommand{\bibjournal}[1]{#1}
	\providecommand{\bibmark}[1]{#1}
	\providecommand{\bibcountry}[1]{#1}
	\providecommand{\bibpatentid}[1]{#1}
	\providecommand{\bibedition}[1]{#1}
	\providecommand{\biborganization}[1]{#1}
	\providecommand{\bibaddress}[1]{#1}
	\providecommand{\bibpublisher}[1]{#1}
	\providecommand{\bibinstitution}[1]{#1}
	\providecommand{\bibschool}[1]{#1}
	\providecommand{\bibvolume}[1]{#1}
	\providecommand{\bibnumber}[1]{#1}
	\providecommand{\bibversion}[1]{#1}
	\providecommand{\bibpages}[1]{#1}
	\providecommand{\bibmodifydate}[1]{#1}
	\providecommand{\bibcitedate}[1]{#1}
	\providecommand{\bibyear}[1]{#1}
	\providecommand{\bibdate}[1]{#1}
	\providecommand{\biburl}[1]{\newline\url{#1}}
近几年来,项目申请人在导师黄勃龙教授指导下,对贵金属应力工程与小分子电催化方面进行了研究,奠定了一定的理论基础。
理论研究基于经过创新改进的第一性原理方法进行计算,申请人与所属课题组成员均为从事第一性原理模拟计算多年的科研工作者,且有丰富的建模与计算经验。并且最近几年,申请人一直致力于对贵金属的晶相结构、各指数面与小分子基电催化的理论研究与计算方法的改进。

在理论准备工作方面,项目申请人对 \ce{Pd3Cu}
的氧还原反应活性进行了第一性原理计算,发现了基于 (100)
、(110)、(111)指数面上各催化位点对氧还原中间体吸附选择性差异,判断(111)指数面将具有优秀的碱性环境氧还原活性[\bibauthor{Wu T\@, Sun M\@, Huang B}\@. \@. \bibjournal{Materials Today
		Energy}\@, \bibvolume{12}\thinspace{}\textnormal{: }\bibpages{426–430}\@,
	\bibyear{2019}\@.],
积累了以催化表面活性位点的几何与电子结构性质开展催化活性分析的宝贵经验。
以此为基础,项目申请人对多种晚期过渡金属的三个低指数面的晶格应力于电子结构的调控进行了深入的理论计算分析,发现了此前同类理论计算研究下,可能因建模时对模型驰豫模拟不足而忽略的于晶格应力正交方向的模型伸缩变化。而该正交于双轴晶格应力的原子间距变化,被
d 带中心数据以及电子态密度分析证明是主导了电子结构变化的因素。
项目申请人进一步向多种过渡金属的高指数面展开研究探索,通过理论推理分析以及参考新近报导的电化学材料制备手段,总结了各高指数面之间的内在几何相关性,提出了从原子尺度对高指数面乃至
fcc/hcp 晶相进行相互转化的理论方法[\bibauthor{Wu T\@, Sun M\@, Wong H~H\@, Huang B}\@. \@. \bibjournal{Nano
		Energy}\@, \bibvolume{85}\thinspace{}\textnormal{: }\bibpages{106026}\@,
	\bibyear{2021}\@.]。
在此基础上,项目申请人还借助分子动力学模拟手段,对过渡金属的高指数面纳米颗粒的表面原子应力进行理论表征,并初步提出了表面位点比例
(surface sites-ratio) 与强应力反差位点密度 (strong-contrast density)
两个判断高指数面的催化活性与稳定性的指标,并结合二氧化碳还原中的关键中间体吸附计算,对高指数面的催化活性来源进行了初步的理论探索[\bibauthor{Wu T\@, Sun M\@, Huang B}\@. \@. \bibjournal{Angewandte Chemie
		International Edition}\@,
	\bibvolume{60}\bibnumber{(42)}\thinspace{}\textnormal{:
	}\bibpages{22996–23001}\@, \bibyear{2021}\@.],为该尚在起步阶段的领域抛砖引玉。
因此,综上所述,项目申请人对于利用第一性原理计算分析材料性质具有一定经验积累,对过渡金属尤其是贵金属体系有较多的经验,同时初步探索了有个人特色的理论分析体系。

在理论计算代码与方法改进方面,项目申请人自本科毕设到博士攻读期间,在导师黄勃龙教授的带领下,针对本项目相关的贵金属材料与小分子基电催化等课题,对密度泛函理论计算代码进行了大量的自定义开发尝试与辅助功能开发。
其中积累的软件开发、自动化流程规划与项目管理能力,为本研究项目提供了坚实的执行基础。

综上所述,本项目在实施前的理论准备基本就绪,在有关小分子基电催化、应力工程和贵金属体系的理论计算工作基础上是扎实的。
