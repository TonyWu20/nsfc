\providecommand{\bibauthor}[1]{#1}
\providecommand{\bibeditor}[1]{#1}
\providecommand{\bibtranslator}[1]{#1}
\providecommand{\bibtitle}[1]{#1}
\providecommand{\bibbooktitle}[1]{#1}
\providecommand{\bibjournal}[1]{#1}
\providecommand{\bibmark}[1]{#1}
\providecommand{\bibcountry}[1]{#1}
\providecommand{\bibpatentid}[1]{#1}
\providecommand{\bibedition}[1]{#1}
\providecommand{\biborganization}[1]{#1}
\providecommand{\bibaddress}[1]{#1}
\providecommand{\bibpublisher}[1]{#1}
\providecommand{\bibinstitution}[1]{#1}
\providecommand{\bibschool}[1]{#1}
\providecommand{\bibvolume}[1]{#1}
\providecommand{\bibnumber}[1]{#1}
\providecommand{\bibversion}[1]{#1}
\providecommand{\bibpages}[1]{#1}
\providecommand{\bibmodifydate}[1]{#1}
\providecommand{\bibcitedate}[1]{#1}
\providecommand{\bibyear}[1]{#1}
\providecommand{\bibdate}[1]{#1}
\providecommand{\biburl}[1]{\newline\url{#1}}
\section{研究内容}

本项目拟基于第一性原理计算与分子动力学模拟手段,构建并整合贵金属催化材料的应力行为数据库。
首先,我们将系统地获取不同应力条件(拉伸、压缩、多轴应变等)下贵金属催化材料(如合金、核壳结构材料)的电子结构参数,(例如 d 带中心、电荷密度分布、功函数等)与吸附能数据。
接下来,我们将整合实验文献中已报道的应力调控催化性能数据(如过电位、质量活性、选择性),建立包含材料组成、应变幅度、活性位点几何构型等信息的结构化数据库。
同时,开发批量建模、数据标注以及检索工具,支持多维度参数关联分析。
比如,可以按照应变类型—吸附能相关性或合金组分—反应选择性等不同组合进行多角度的数据透视与分析。
数据库初步建立后,我们将结合机器学习算法,挖掘数据库中海量数据的内在关联,建立应力作用与小分子基电化学反应催化性能的定量预测模型,进一步发展以 d 带中心理论为基础的分析体系。
与此同时,还可以结合以密度泛函理论(DFT)为代表的第一性原理计算数据,训练机器学习模型直接预测吸附能计算结果,加速后续数据集的生成和准确度。
我们还将基于数据库收录的信息,通过系统替换贵金属元素及模拟晶界、空位等缺陷工程,构建虚拟材料库。
由此,我们可以通过高通量计算模式,模拟大量同类结构材料受应力调控后的催化活性与材料性质,筛选潜力候选材料。
另外,我们还将结合第一性原理计算与分子动力学模拟手段,研究应力对关键中间体吸附能以及反应路径选择性的影响,建立应变—选择性关联规则,指导设计高选择性催化剂。


\section{研究目标}

\begin{enumerate}
	\item
	      构建首个面向小分子电催化的贵金属应力行为数据库。
	      该数据库将从二维纳米贵金属催化材料与二氧化碳还原的单碳(C1)与双碳(C2)产物路径出发,逐步涵盖主流贵金属体系及典型小分子电催化反应(电解水、二氧化碳还原、氮还原等),提供电子结构、吸附能、动力学参数的标准化数据资源;

	\item
	      开发数据驱动的应力效应建模与预测工具。
	      通过编程开发自动化批量工作流软件,进行高通量系统化的贵金属材料建模,并结合机器学习发掘数据内在关联性,实现从应变参数输入到催化性能输出的定量预测,突破``试错式''与``经验先导式''等传统研究范式的效率与成本瓶颈。

	\item
	      通过数据库支持的材料筛选与反应路径优化,深化应力作用调控贵金属电催化剂设计的理论模型,结合先进实验表征技术,设计与筛选出兼具高活性、高选择性及稳定性的贵金属电催化剂。
\end{enumerate}
\section{拟解决的关键科学问题}
\subsection{应力作用与催化性能的量化关联机制}

d 带中心(d-band center)是理论计算领域中描述过渡金属的电子结构特征的一个重要参数,反映了 d 带电子态的平均能级水平,可以定性解释并预测过渡金属表面的吸附强度。
d 带中心的定义如下:
$$
\setlength{\abovedisplayskip}{12pt}
\setlength{\belowdisplayskip}{12pt}
	\varepsilon_d = \frac{\int \rho EdE}{\int \rho dE}
$$

其中,$\varepsilon_{d}$ 即 d 带中心,$E$ 为 d
带电子态对应的能量,$\rho$ 为 d 带电子态密度(Density of States, DOS)。

由此可见,d 带中心的计算与电子态密度的取样范围(整个表面或仅反应位点原子)密不可分, 同时也存在通过考虑原子 d 轨道的空间分量而进一步细化 d 带中心
计算的可能。
例如,当吸附物锚定于位点的上方时,应主要考虑
$d_{xz}$ 、$d_{yz}$ 、$d_{z^{2}}$ 的贡献。
同样地,当应力作用于材料表面时,原子间距的改变也受应力来源的方向主导。
例如, $xy$
方向的双轴应力作用,应当主要拉伸或压缩同一原子层内的原子,直接改变 $xy$ 平面上的原子间距。
而材料本身在 $xy$  方向受到外加应力作用后,为了平衡体系能量的变化,将会在正交于应力方向的 $z$ 轴上产生类似泊松效应的自发形变,从而也改变了 $z$ 轴方向的原子间距。
我们在本研究的前期准备中,通过 DFT 计算模拟了晚期过渡金属(Fe、Co、Ni、Pd、Pt)的(100),
(110)与(111)晶面上的双轴应力作用。
其中,我们发现在(100) 与
(110)上,泊松效应带来的 z
轴方向的原子层间距变化,反而主导了表面的 d 带中心的变化趋势[\bibauthor{Wu T\@, Sun M\@, Huang B}\@. \@. \bibjournal{Small}\@,
\bibvolume{16}\bibnumber{(38)}\@, \bibyear{2020}\@.]。
如前所述,作用于 $xy$ 方向的拉伸应力将诱发 $z$ 方向的自发压缩,以稳定体系的能量状态。
此时,(110) 与(100)的晶面模型中,d 带中含 $z$ 方向的电子态重叠增强,且对 d 带中心能级变化的贡献更多,因此拉伸应力作用反而使这些晚期过渡金属的 (100) 与 (110) 的整体 d 带中心能级上升。
此研究表明,对应力调控电子结构的机制解释上,d 带中心理论依然有进一步深化应用的广阔空间。

另外,当前 d 带中心理论在解释复杂应力体系(如多轴应变、缺陷诱导应变)的吸附能变化时,往往在建模环节采用了较小的超晶胞 (supercell) 尺寸,或者将大部分基底原子固定,这种设置无意中屏蔽了吸附物对活性表面带来的应力变化[\bibauthor{Francis M~F\@, Curtin W~A}\@. \@. \bibjournal{Nature Communications}\@, \bibvolume{6}\bibnumber{(1)}\@, \bibyear{2015}\@.],而该局域应力作用带来的电子结构变化不可被轻易忽视。
因此,这类过于简单的计算模拟设置,不利于正确分析与预测实际催化过程中催化表面的应力分布情况与相应的电子结构活性变化。
最后,当前对应力作用调控贵金属催化材料表面活性的理论计算研究,往往停留在对实验结果的后验分析。
而从理论出发,推导具体应如何应用应力作用于给定的贵金属催化材料达到目标活性的工作,目前依然处于起步阶段。
究其原因,正是因为应力对贵金属催化材料的理化性质调控涉及多方面的作用机制,所以难以通过单纯考虑电子结构或活性位点几何状态的方法进行预测。
因此,需通过数据库建立多参数模型(如应变幅度、配位数、电荷转移),探索各项参数与催化活性的定量关系。

\subsection{贵金属材料应力响应的高效筛选策略}

当前应力调控研究多依赖于``试错式''实验探索,但此类方法难以系统拓展至多元合金或复杂缺陷体系,效率低下且成本高昂。
本数据库的构建将重点解决以下瓶颈:

\begin{enumerate}
	\item
	      \textbf{理论驱动的材料组合预测}:基于第一性原理计算方法,系统模拟不同贵金属元素(如Pd、Ru、Ir)及其合金在应变作用下的电子结构响应,构建``元素组合-应力响应''映射关系。
		  我们可以从实验报道的材料为模板模型,进而生成大量不同贵金属元素而结构相似的模型。
		  例如,通过获取 PtNi 合金的拉伸应变数据[\bibauthor{Zhao X\@, Xi C\@, Zhang R\@, Song L\@, Wang C\@, Spendelow J~S\@,
		      Frenkel A~I\@, Yang J\@, Xin H~L\@, Sasaki K}\@. \@. \bibjournal{ACS
		      Catalysis}\@, \bibvolume{10}\bibnumber{(18)}\thinspace{}\textnormal{:
	      }\bibpages{10637–10645}\@, \bibyear{2020}\@.],预测 PdFe 、IrNi 或 RuAu 体系催化活性对外加应力的敏感性,指导实验优先验证高潜力候选材料。
	\item
	      \textbf{缺陷工程的高通量模拟}:针对晶界、位错等缺陷诱导的局部应变,开发自动化计算流程,评估不同缺陷类型与密度对催化活性的影响规律。
	      譬如,结合晶界应变的实验数据[\bibauthor{Wu G\@, Han X\@, Cai J\@, Yin P\@, Cui P\@, Zheng X\@, Li H\@, Chen
		      C\@, Wang G\@, Hong X}\@. \@. \bibjournal{Nature Communications}\@,
	      \bibvolume{13}\bibnumber{(1)}\@, \bibyear{2022}\@.],模拟高指数晶面缺陷对还原选择性的调控作用。
	\item
	      \textbf{机器学习辅助的逆向设计}:利用数据库中的海量数据训练生成对抗网络(GAN),生成满足目标性能(如HER过电位\textless50
	      mV)的虚拟材料结构(如特定应变幅度的核壳纳米线),突破传统试错模式的局限性。
\end{enumerate}

\subsection{应变对催化选择性的动态调控规律}
在复杂电催化反应(如二氧化碳还原、氮还原)中,应力对选择性的调控虽有实验方面的报道[\bibauthor{Wei W\@, Guo F\@, Wang C\@, Wang L\@, Sheng Z\@, Wu X\@, Cai B\@,
	Eychm\"{u}ller A}\@. \@. \bibjournal{Small}\@,
	\bibvolume{20}\bibnumber{(25)}\@, \bibyear{2024}\@.; \bibauthor{Hao J\@, Zhuang Z\@, Hao J\@, Cao K\@, Hu Y\@, Wu W\@, Lu S\@, Wang
		C\@, Zhang N\@, Wang D\@, Du M\@, Zhu H}\@. \@. \bibjournal{ACS Nano}\@,
	\bibvolume{16}\bibnumber{(2)}\thinspace{}\textnormal{:
	}\bibpages{3251–3263}\@, \bibyear{2022}\@.],但机制尚未明晰。
%    以蔡彬团队2021年的研究为例,Ru-Au气凝胶的微应变不仅提升了HER活性,还通过调控
% \ce{ ^*H } 与 \ce{^*COOH}
%    中间体的竞争吸附路径,显著提高了甲酸选择性\cite{Wei_2024}。
%    而朱罕团队也报道了 PdNi
%    合金通过应力精准的调控,获得了高达96.6\%的一氧化碳转化率\cite{Hao_2022}。
%    然而,此类现象缺乏普适性理论解释,且难以预测。
因此,本数据库将对应力对催化选择性的调控从理论计算角度进行初探,将聚焦以下核心问题:

\begin{enumerate}
	\item
	      \textbf{应变-吸附路径关联建模}:通过分子动力学模拟手段,量化应变对关键中间体(如二氧化碳还原中的\ce{^{*}CO}、\ce{^*OCHO})吸附构型的影响,建立应变幅度与吸附位点偏好性的定量关系。
		  例如,拉伸应变可能扩大Pt-Pt间距,抑制\ce{^*CO}的桥位吸附,从而促进C-C键断裂[\bibauthor{Miao B-Q\@, Yuan Z-H\@, Liu X-L\@, Ai X\@, Zhao G-T\@, Chen P\@, Jin
		      P-J\@, Chen Y}\@. \@. \bibjournal{Chinese Journal of Chemistry}\@,
	      \bibvolume{42}\bibnumber{(21)}\thinspace{}\textnormal{:
	      }\bibpages{2633–2640}\@, \bibyear{2024}\@.]。
	\item
	      \textbf{选择性调控的电子-几何协同机制}:结合电子结构参数( d 带中心、电荷转移量)与几何参数(键长、表面曲率),解析应变对反应路径分支点的调控作用。
		  例如,在\ce{Pd3Pb}纳米线中,拉伸应变通过增强Pd-Pd轨道杂化,选择性促进乙醇氧化的C-C键断裂而非脱氢路径[\bibauthor{Miao B-Q\@, Yuan Z-H\@, Liu X-L\@, Ai X\@, Zhao G-T\@, Chen P\@, Jin
		      P-J\@, Chen Y}\@. \@. \bibjournal{Chinese Journal of Chemistry}\@,
	      \bibvolume{42}\bibnumber{(21)}\thinspace{}\textnormal{:
	      }\bibpages{2633–2640}\@, \bibyear{2024}\@.]。
	\item
	      \textbf{多反应耦合场景下的动态优化}:针对碳中和相关电催化反应中的小分子转化,建立应变作用下多中间体竞争吸附的动力学模型,预测最优应变条件以实现目标产物(如尿素或氨气)的高选择性合成。
\end{enumerate}

\section{本项目特色与创新点}

\begin{enumerate}
	\item \textbf{数据驱动的应力效应量化模型}

	      突破传统 d 带中心理论的单一框架,首次构建融合应变参数、电子结构及几何构型的多变量预测模型,通过机器学习解析复杂应力体系的非线性响应规律,实现催化性能的精准预测。
	\item \textbf{高通量虚拟筛选与实验闭环验证}

	      基于数据库开发``理论计算——机器学习——实验验证''全链条研发范式,系统性覆盖多元合金、缺陷工程等复杂体系,大幅提升新型应力催化剂的开发效率。

	\item \textbf{应变-选择性动态调控机制解析}

	      揭示应力对多反应路径分支点的动态调控规律,结合分子动力学与电子结构分析,为碳中和相关小分子电化反应设计高选择性催化剂提供理论工具。
\end{enumerate}

\section{年度研究计划}
本项目的年度研究计划如下:

\textbf{2027年1月至12月:}

对文献进行调研,回顾近年来其他理论课题组和本课题组对于贵金属应力调控分析的理论研究工作,总结贵金属材料体系中应力工程的应用特点与理论计算建模方法,以及有关方向的理论基础和研究现状。

开发贵金属催化材料批量修改与生成和反应中间体布置于各式吸附位点的软件代码,理顺计算参数输入------计算任务执行------计算数据整理的自动化链条,为后续的高通量计算工作打下坚实的基础。
同时,学习摸索利用 GPU
集群加速第一性原理计算的方式,促进数据库所需大量数据的高效生成。

\textbf{2028年1月至12月:}

在所需的软件代码开发与调试完成后,开始根据文献调研中所得的实验数据与总结所得的理论建模方法,对二维贵金属纳米材料的应力调控进行建模计算。将所得的计算结果进行分析,对照其文献记载的活性调控结果,分析电子结构、材料结构等多维度的变量,探索应力增强的活性来源。
其中,重点关注各向异性应力以及潜在的电荷转移现象对电子结构的影响,理清应力的机械作用与其他电子转移作用对催化活性调控的贡献占比。

\textbf{2029年1月至12月:}

结合已有的理论计算与结论,开始根据数据库已有数据集生成虚拟材料模型,利用机器学习方法,探索并预测应力作用中各种相关变量对贵金属催化活性的影响。
同时,在各种材料模型上进行二氧化碳还原反应的相关中间体的吸附模拟计算,定量分析应力作用对反应活性与选择性的调控。
建立友好的数据库查阅界面,将数据库向各实验课题组开放使用,联系紧密合作的课题组团队,对数据库预测的高活性材料进行实验合成与活性验证。参加国内外重要会议,在国际会议上报导我们的研究成果,完成结题报告,上报结题。

\section{预期研究结果}
本项目预期取得如下研究结果:
\begin{enumerate}
	\item
	      获得大量的高质量高精度贵金属应力下催化行为变化的理论计算数据;
	\item
	      建立初步的预测应力调控催化活性效果的定量分析模型;
	\item
	      揭示在贵金属二维材料上,各种性质的应力对催化活性与选择性的调控机制;
	\item
	      对开发高选择性高活性小分子基电催化贵金属催化剂提供理论指导;
	\item
	      在国内外著名学术刊物上发表高水平论文5篇以上。
\end{enumerate}
针对项目所提问题,作出自己特色,努力使课题组在该领域的研究达到世界前沿水平。研究成果以论文和数据库网站形式发表。



